\documentclass[english]{amsart}
\usepackage{amsmath}
\usepackage{amssymb}
\usepackage[all]{xy}
\usepackage{xypic,amsthm,amssymb,hyperref,amsmath,graphicx,stmaryrd,boxedminipage,mathrsfs,fullpage,manfnt}
\usepackage[all]{xy}
\usepackage[T1]{fontenc}
\usepackage[sc]{mathpazo}
\usepackage[mathcal]{euscript}
\usepackage{calligra}
\usepackage{sseq}
\newcommand{\R}{\mathbf{R}}
\newcommand{\C}{\mathbf{C}}
\newcommand{\Z}{\mathbf{Z}}
\newcommand{\Q}{\mathbf{Q}}
\renewcommand{\P}{\mathbb{P}}
\newcommand{\sma}{\wedge}
\newcommand{\ad}{\operatorname{ad}}
\newcommand{\Ad}{\operatorname{Ad}}
\newcommand{\sgn}{\operatorname{sgn}}
\newcommand{\ind}{\operatorname{ind}}
\newcommand{\ti}{\;\;\makebox[0pt]{$\top$}\makebox[0pt]{$\cap$}\;\;}
\newcommand{\codim}{\operatorname{codim}}
\newcommand{\GL}{\operatorname{GL}}
\newcommand{\gl}{\mathfrak{gl}}
\newcommand{\SL}{\operatorname{SL}}
\renewcommand{\sl}{\mathfrak{sl}}
\renewcommand{\o}{\mathfrak{o}}
\newcommand{\SO}{\operatorname{SO}}
\newcommand{\so}{\mathfrak{so}}
\newcommand{\Sp}{\operatorname{Sp}}
\newcommand{\symp}{\mathfrak{sp}}
\newcommand{\SU}{\operatorname{SU}}
\newcommand{\su}{\mathfrak{su}}
\newcommand{\ext}{\operatorname{Ext}}
\newcommand{\tor}{\operatorname{Tor}}
\renewcommand{\hom}{\operatorname{Hom}}
\newcommand{\im}{\operatorname{Im}}
\newcommand{\coker}{\operatorname{coker}}
\newcommand{\obj}{\operatorname{obj}}
\newcommand{\id}{\operatorname{Id}}
\newcommand{\st}{\operatorname{st}}
\newcommand{\tr}{\operatorname{Tr}}
\newcommand{\diam}{\operatorname{diam}}
\newcommand{\Spec}{\operatorname{Spec}}
\newcommand{\aut}{\operatorname{Aut}}
\newcommand{\syl}{\operatorname{Syl}}
\newcommand{\var}{\operatorname{Var}}
\newcommand{\ann}{\operatorname{Ann}}
\newcommand{\gal}{\operatorname{Gal}}
\newcommand{\nil}{\operatorname{nil}}
\newcommand{\map}{\operatorname{Map}}
\newcommand{\res}{\operatorname{Res}}
\newcommand{\re}{\operatorname{Re}}
\newcommand{\rel}{\operatorname{rel}}
\newcommand{\vect}{\operatorname{Vect}}
\newcommand{\boxit}[1]{\begin{boxedminipage}{13cm} #1 \end{boxedminipage}}
\newcommand{\triv}{\operatorname{Triv}}
\newcommand{\Sq}{\operatorname{Sq}}
\newcommand{\sq}{\operatorname{Sq}}
\newcommand{\aind}{\operatorname{a-ind}}
\newcommand{\tind}{\operatorname{t-ind}}
\newcommand{\mo}{\mathbf{MO}}
\newcommand{\ch}{\operatorname{ch}}
\newcommand{\td}{\operatorname{Td}}
\newcommand{\ahat}{\widehat{A}}
\newcommand{\calC}{\mathcal{C}}
\newcommand{\calD}{\mathcal{D}}
\newcommand{\colim}{\operatorname{colim}}
\newcommand{\weakequiv}{\xrightarrow{\sim}}
\newcommand{\cofib}{\hookrightarrow}
\newcommand{\fib}{\twoheadrightarrow}
\newcommand{\xycofib}{^{(}->}
\newcommand{\xyfib}{->>}
\newcommand{\Ho}{\operatorname{Ho}}
\newcommand{\holim}{\operatorname{holim}}
\newcommand{\hocolim}{\operatorname{hocolim}}
\newcommand{\mf}{\mathfrak}
\newcommand{\Shom}{\text{\calligra Hom}}
\renewcommand{\div}{\operatorname{div}}
\newcommand{\Weil}{\operatorname{Weil}}
\newcommand{\Cl}{\operatorname{Cl}}
\newcommand{\Pic}{\operatorname{Pic}}
\newcommand{\Proj}{\operatorname{Proj}}
\newcommand{\fun}{\operatorname{Fun}}
\newcommand{\mc}{\mathcal}
\newcommand{\hh}{\operatorname{HH}}
\newcommand{\mbf}{\mathbf}
\newcommand{\thh}{\operatorname{THH}}
\newcommand{\diag}{\operatorname{diag}}
\newcommand{\spsh}{\operatorname{sPSh}}
\newcommand{\Map}{\operatorname{Map}}
\newcommand{\EP}{\widetilde{E}\mathcal{P}}
\newcommand{\height}{\operatorname{ht}}
\newcommand{\Gal}{\operatorname{Gal}}
\newcommand{\spf}{\operatorname{Spf}}
\newcommand{\cart}{\operatorname{Cart}}
\newcommand{\Cart}{\text{Cart}}
\newcommand{\W}{\mathbf{W}}
\newcommand{\ques}[1]{\textcolor{red}{#1}}
\newtheorem*{thm}{Theorem}
\newtheorem*{lem}{Lemma}
\newtheorem*{cor}{Corollary}
\newtheorem*{prop}{Proposition}
\newtheorem*{conj}{Conjecture}
\newtheorem*{proto}{Proto-Theorem}
\newtheorem*{claim}{Claim}


\theoremstyle{definition}
\newtheorem*{defn}{Definition}
\newtheorem*{example}{Example}
\newtheorem*{rmk}{Remark}
\newtheorem*{nota}{Notation}
\newtheorem*{exercise}{Exercise}
\newtheorem*{fact}{Fact}


\title{Etale Cohomology}
\author{Jonathan Campbell}
\begin{document}
\maketitle
\tableofcontents 


\section{Basic Stuff I should Have Known / Need to Remember}

Completions - 


Separable
\begin{defn}
A field extension is \textbf{separable} if it's minimal polynomial is separable (i.e. no repeated roots, i.e. $f'(a) \neq 0$ for root $a$)
\end{defn}

The 100 forms of Nakayama's Lemma

Topological aspects of schemes that always confused me. 


Ring extensions

Differentials


\section{Local Theory}

\subsection{Flatness}

For something with so many consequences I find it useful to separate into a few different cases. For example, I like to consider
\begin{itemize}
\item Characterization
\item Properties
\item Consequences
\item Special Cases / Examples 
\end{itemize}

Start with \textbf{characterizations} of flat morphisms (of rings). After that, it's flat modules. 

\begin{thm}
A morphism $A \to B$ is flat if any of the following hold
\begin{itemize}
\item The functor $M \mapsto B \otimes_A M$ is exact
\item For all ideals $I \subset A$, the homomoprhism $I \otimes_A B \to B$ given by $a \otimes b \mapsto f(a) b$ is injective. 
\end{itemize}
\end{thm}

Some \textbf{properties} are not so hard to come by 
\begin{prop}
Flatness plays well with localization. 
\end{prop}

\begin{rmk}
In practice, this means that flatness can be checked locally. That is, $f: A\to B$ is flat if for all $\mf{n} \subset B$, $A_{f^{-1}(\mf{n})} \to B_{\mf{n}}$ is flat. That is, usually flatness only has to be check on local rings. 
\end{rmk}

Other properties that are key

\begin{prop}-
\begin{enumerate}
\item Open immersions are flat
\item Flat morphisms compose
\item They pullback. 
\end{enumerate}
\end{prop}


\textbf{Special Cases}

\begin{enumerate}
\item For $A$ an integral domain $A \to B$ is flat implies $A \to B$ injective. 
\item For $A$ a Dedekind domain $A \to B$ is flat iff $A \to B$ is injective. 
\end{enumerate}


\textbf{Flat Modules}

There are a bunch of equivalent characterizations for flat modules

\begin{thm}
Let $M$ be a finitely generated $A$-module. TFAE
\begin{enumerate}
\item $M$ is flat
\item $M_{\mf{m}}$ is free as an $A_{\mf{m}}$ module
\item $\widetilde{M}$ is a locally free sheaf on $\Spec A$
\item $M$ is projective. 
\item (Special case) For $A$ an integral domain, then everything is equiv to $\dim_{k(\mf{p})} (M\otimes_A k(\mf{p}))$ is the same for all primes $\mf{p}$. 
\end{enumerate}
\end{thm}

The algebraic notion of flatness is used to define the geometric notion. 



\subsection{Ramified-Ness}

\ques{I guess this is supposed to be something in number theory, right? Where is it useful? }

here's the ring-theoretic notion;

\begin{defn}
A local morphism $A \to B$ is \textbf{unramified} if $B/f(\mf{m}_a) B$ is a finite separable field extension of $A/\mf{m}_A$. 
\end{defn}

\begin{rmk}
A finite field extension is ramified iff it is separable. 
\end{rmk}


For schemes: The notion of ramified is defined for locally finite type maps. 

\begin{defn}
A locally finite type morphism of schemes $X \to Y$ is \textbf{unramified} if at $y \in Y$ if $\mc{O}_{Y,y}/ \mf{m}_x \mc{O}_{X,x}$ is a finite separable field extension of $k(x)$. 
\end{defn}

\begin{rmk}
A morphism $X \to Y$ is formally unramified if $\Omega_{X/Y} = 0$. When $X \to Y$ is locally of finite type this is equivalent to the above definition. 
\end{rmk}

We finally have characterizations of unramified morphisms

\begin{thm}
$Y \to X$ locally of finite type. 
\begin{enumerate}
\item $f$ unramified
\item For all $x \in X$, $Y_x \to \Spec k(x)$ unramified
\item All geometric fibers of $f$ are unramified
\item For all $x \in X$, $Y_x \cong \coprod \Speck k_i$ where $k_i$ are finite separable field extensions of $k(x)$
\item The sheaf $\Omega^1_{X/Y} = 0$
\item The diagonal morphism $\Delta_{Y/X} : Y \to Y \times_X Y$ is an open immersion. 
\item There is an open neighborhood $U \ni x$ and $V \ni Y$ of $f(x)$ an da diagram
\[
\xymatrix{
U \ar[rr]^i \ar[dr] & & \mbf{A}^n_V\\
 & V & 
}
\]
where $i$ is a closed immersion defined by $\mc{I}$ such that the differentials for $dg$ for $g \in \mc{I}_{i(x)}$ generate $\Omega_{\mbf{A}^n_V/V,i(x)}$. 
\end{enumerate}
\end{thm}
\begin{proof}

\end{proof}

\subsubsection{Examples}

\begin{example}
$k[y] \to k[x]$, $y \mapsto x^2$. 
\end{example}

\begin{example}
There is a big source of unramified morphisms coming from affine maps
\[
\Spec (R[x_1, \dots, x_n]/I) \to \Spec(R)
\]
that satisfy certain properties. Suppose $I$ is generated by $(g_1, \dots, g_m)$ and $\mf{q} \subset R[x_1, \dots, x_n]$ prime corresponding to a point we'll call $x$. If we assume $I \subset \mf{q}$, i.e. the point is in the closed subscheme, and consider the matrix
\[
\left( \frac{\partial g_i}{\partial x_j} \right) \ \text{mod}\ \mf{q} 
\]
then $\Spec(R[x_1, \dots, x_n]/I) \to \Spec (R)$ is unramified at $x$ if the matrix has rank $n$. 

\end{example}



\subsection{Etale-Ness}

\begin{defn}
\textbf{Etale} means smooth and unramified. 
\end{defn}

\textbf{standard etale} presentation

\subsection{Henselian Rings}

\begin{defn}
A \textbf{Henselian (local) ring} ...
\end{defn}

\begin{thm}[Characterizations of Henselian]

\end{thm}

\begin{prop}
Any complete local ring is Henselian. 
\end{prop}

\begin{defn}
\textbf{Strictly Henselian} or \textbf{strictly local}
\end{defn}


\section{Sites and Sheaves on the Etale Site}

\begin{defn}
\textbf{Grothendieck topology}
\end{defn}


\begin{defn}
\textbf{Sheaf condition}
\end{defn}


\begin{example}
Etale sheaf over $\operatorname{Spec}(k)$. 
\end{example}

Checking the sheaf condition for the etale topology

\begin{prop}
Let $\mc{F}$ be a presheaf on $X_{\text{et}}$. It suffices to check that $\mc{F}$ is a sheaf in the Zariski topology and $\mc{F}$ is a sheaf for etale coverings $V \to U$ with $V, U$ both affine, i.e. 
\[
\mc{F}(V) \to \mc{F}(U) \to \mc{F}(V \times_U V)
\]
is exact. 
\end{prop}


\subsection{Direct and Inverse Limits of Sheaves}









\end{document}
